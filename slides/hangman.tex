\documentclass[spanish]{beamer}

\usetheme{default}

\setbeamertemplate{navigation symbols}{}

\usepackage{polyglossia}

\setdefaultlanguage{spanish}
\setotherlanguage[variant=usmax]{english}

\usepackage{fancyvrb}

\DefineShortVerb{\|}
\DefineVerbatimEnvironment{code}{Verbatim}{frame=lines}

\title{Ahorcado}
\subtitle{Programación funcional imperativa}
\author{Juan Pedro Villa Isaza}
\institute{Stack Builders}
\date{24 de abril de 2015}

\logo{\includegraphics[height=0.5cm]{stackbuilders.png}}

\begin{document}

%%%%%%%%%%%%%%%%%%%%%%%%%%%%%%%%%%%%%%%%%%%%%%%%%%%%%%%%%%%%%%%%%%%%%%

\frame{\titlepage}

%%%%%%%%%%%%%%%%%%%%%%%%%%%%%%%%%%%%%%%%%%%%%%%%%%%%%%%%%%%%%%%%%%%%%%

\begin{frame}
  \begin{columns}[onlytextwidth,T]
    \begin{column}{.5\textwidth}
      \begin{center}
        \includegraphics[scale=0.8]{haskell.png}
      \end{center}
    \end{column}
    \begin{column}{.5\textwidth}
      Haskell es:
      \begin{itemize}
      \item<1-> Funcional
      \item<1-> \emph{Puro}
      \item<1-> ...
      \item<2-> \emph{Interesante}
      \item<3-> ¿\alert{Inútil}?
      \end{itemize}
    \end{column}
  \end{columns}
\end{frame}

%%%%%%%%%%%%%%%%%%%%%%%%%%%%%%%%%%%%%%%%%%%%%%%%%%%%%%%%%%%%%%%%%%%%%%

\begin{frame}[fragile]
  \begin{itemize}
  \item
    \begin{code}
reverse :: [a] -> [a]
reverse = ...
    \end{code}
  \item
    \begin{code}
> reverse "haskell"
"lleksah"
    \end{code}
  \end{itemize}
\end{frame}

%%%%%%%%%%%%%%%%%%%%%%%%%%%%%%%%%%%%%%%%%%%%%%%%%%%%%%%%%%%%%%%%%%%%%%

\begin{frame}[fragile]
  \begin{itemize}
  \item
    \begin{code}
main :: IO ()
main = putStrLn (reverse "haskell")
    \end{code}
  \item
    \begin{code}
> main
lleksah
    \end{code}
  \end{itemize}
\end{frame}

%%%%%%%%%%%%%%%%%%%%%%%%%%%%%%%%%%%%%%%%%%%%%%%%%%%%%%%%%%%%%%%%%%%%%%

\begin{frame}[fragile]
  \begin{itemize}
  \item
    \begin{code}
main :: IO ()
main = do
  line <- getLine
  putStrLn (reverse line)
    \end{code}
  \item
    \begin{code}
> main
haskell
lleksah
    \end{code}
  \end{itemize}
\end{frame}

%%%%%%%%%%%%%%%%%%%%%%%%%%%%%%%%%%%%%%%%%%%%%%%%%%%%%%%%%%%%%%%%%%%%%%

\begin{frame}[fragile]
  \begin{center}
    |type IO a = Mundo -> (a,Mundo)|
  \end{center}
  \begin{center}
    \includegraphics[scale=0.3]{hatter}
  \end{center}
\end{frame}

%%%%%%%%%%%%%%%%%%%%%%%%%%%%%%%%%%%%%%%%%%%%%%%%%%%%%%%%%%%%%%%%%%%%%%

\begin{frame}
  \begin{quote}
    Haskell es el mejor lenguaje de programación imperativa del mundo.
  \end{quote}
  \hfill---Simon Peyton Jones
\end{frame}

%%%%%%%%%%%%%%%%%%%%%%%%%%%%%%%%%%%%%%%%%%%%%%%%%%%%%%%%%%%%%%%%%%%%%%

\begin{frame}
  \frametitle{Ahorcado}

\end{frame}

%%%%%%%%%%%%%%%%%%%%%%%%%%%%%%%%%%%%%%%%%%%%%%%%%%%%%%%%%%%%%%%%%%%%%%

\begin{frame}
  \begin{itemize}
  \item
    Distinción clara entre:
    \begin{itemize}
    \item Acciones y funciones
    \item Código impuro y código impuro
    \item Programación imperativa y programación no imperativa
    \end{itemize}
  \item
    Razonamiento ecuacional
  \item
    Refactoring
  \item
    Menos dolores de cabeza
  \end{itemize}
\end{frame}

%%%%%%%%%%%%%%%%%%%%%%%%%%%%%%%%%%%%%%%%%%%%%%%%%%%%%%%%%%%%%%%%%%%%%%

\begin{frame}
  \begin{quote}
    Haskell es el mejor lenguaje de programación imperativa del mundo.
  \end{quote}
  \hfill---Simon Peyton Jones
\end{frame}

%%%%%%%%%%%%%%%%%%%%%%%%%%%%%%%%%%%%%%%%%%%%%%%%%%%%%%%%%%%%%%%%%%%%%%

\begin{frame}
  Con base en las respuestas de más de 368000 personas:

  El aprendizaje de este lenguaje cambió significativamente la menera
  en que uso otros lenguajes.
  \begin{columns}[onlytextwidth,T]
    \begin{column}{.5\textwidth}
      \begin{enumerate}
      \item Haskell
      \item Scheme
      \item Coq
      \item Common Lisp
      \item Erlang
      \end{enumerate}
    \end{column}
    \begin{column}{.5\textwidth}
      \begin{enumerate}
      \setcounter{enumi}{42}
      \item Fortran
      \item R
      \item AWK
      \item Visual Basic
      \item Cobol
      \end{enumerate}
    \end{column}
  \end{columns}
\end{frame}

%%El aprendizaje de este lenguaje mejoró mi habilidad como programador.

%% Learning this language improved my ability as a programmer
%% (http://hmrp.pl/zdDNYP). 1. Haskell 2. Standard ML 3. Scheme 4.
%% Coq 5. Common Lisp


\begin{frame}
  Recomendaría a la mayoría de programadores aprender este lenguaje,
  sin importar si tienen o no una necesidad específica de hacerlo.

  I would recommend most programmers learn this language, regardless
  of whether they have a specific need for it (http://hmrp.pl/zuMkvG).
  1. Haskell 2. Coq 3. Agda 4. Clojure 5. Scheme
\end{frame}

%%%%%%%%%%%%%%%%%%%%%%%%%%%%%%%%%%%%%%%%%%%%%%%%%%%%%%%%%%%%%%%%%%%%%%

\begin{frame}
  \begin{center}
    \url{https://github.com/stackbuilders/hangman}
  \end{center}
\end{frame}

%%%%%%%%%%%%%%%%%%%%%%%%%%%%%%%%%%%%%%%%%%%%%%%%%%%%%%%%%%%%%%%%%%%%%%

\end{document}
